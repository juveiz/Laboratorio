\label{prologo}
La idea detr\'as de la existencia de un curso como este es f\'acil de expresar:
{\itshape es muy conveniente que alumnos de un Grado en Ciencias Matem\'aticas
tengan la oportunidad, desde
el comienzo de sus estudios, de conocer y practicar una aproximaci\'on
experimental al estudio de las matem\'aticas.}

\smallskip

La idea de las matem\'aticas como una ciencia experimental aparece despu\'es de
que se generalizara el uso de los ordenadores personales, en nuestro pa\'{\i}s a
partir de los a\~nos $90$, y, sobre todo,  con la aparici\'on de ordenadores
personales con suficiente potencia de c\'alculo ya en este siglo. 

\smallskip

Sin embargo, algunos matem\'aticos de siglos anteriores demostraron una
extraordinaria capacidad de c\'alculo mental y realizaron sin duda experimentos
que les convencieron de que las ideas que pudieran tener sobre un determinado
problema eran correctas, o bien de que eran  incorrectas. Un ejemplo claro es el
de Gauss en sus trabajos sobre teor\'{\i}a de n\'umeros: se sabe, ya que se
conservan los cuadernos que utilizaba, que fundamentaba sus afirmaciones en
c\'alculos muy extensos de ejemplos concretos.

\smallskip

Los ordenadores personales extienden nuestra capacidad de c\'alculo, en estos
tiempos bastante limitada, y permiten realizar b\'usquedas exhaustivas de
ejemplos o contraejemplos. Podemos decir, hablando en general, que los
ordenadores realizan para nosotros, de forma muy eficiente,  tareas repetitivas
como la ejecuci\'on, miles de veces,  de un bloque de instrucciones que por
alg\'un motivo nos interesa.  

\smallskip

Es cierto que el ordenador no va a demostrar un teorema por nosotros, pero puede
encontrar un contraejemplo, probando as\'{\i} que el resultado propuesto es
falso, y tambi\'en puede, en ocasiones, ayudarnos a lo largo del desarrollo de
una demostraci\'on. Sobre todo nos convencen de que una cierta afirmaci\'on es
plausible y, por tanto, puede merecer la pena pensar sobre ella.


\bigskip


\pagebreak[3]

{\sc Descripci\'on del curso}

En este curso usamos {\sage} como instrumento para calcular y programar. Puedes
ver una descripci\'on de la forma de acceder a su uso en el primer cap\'{\i}tulo de estas notas. 


Comenzamos estudiando el uso de {\sage} como calculadora avanzada, es decir, su
uso
para obtener respuesta inmediata a nuestras preguntas mediante programas ya
incluidos dentro del sistema. Por ejemplo, mediante {\tt factor(n)} podemos
obtener la descomposici\'on como producto de factores primos de un entero {\tt
n}. 

{\sage} tiene miles de tales instrucciones preprogramadas que resuelven muchos
de
los problemas con los que nos podemos encontrar. En particular, revisaremos las
instrucciones para resolver problemas de c\'alculo y \'algebra lineal, as\'{\i}
como las instrucciones para obtener representaciones gr\'aficas en $2$ y $3$
dimensiones.


A continuaci\'on, despu\'es describir las estructuras de datos y su manipulaci\'on,  de  trataremos los aspectos b\'asicos de la programaci\'on usando
el lenguaje Python, que es el lenguaje utilizado en una gran parte de
{\sage}, y es el que usaremos a lo largo del curso.
Concretamente, veremos los bucles {\tt for} y {\tt while}, el control del flujo
usando {\tt if} y {\tt else}, y la recursi\'on. Con estos pocos mimbres haremos
todos los cestos que podamos durante este curso. 

Por \'ultimo, el curso contiene cuatro  bloques, aproximaci\'on, aritm\'etica,
criptograf\'{\i}a y
teor\'{\i}a de la probabilidad, que desarrollamos todos los cursos junto con uno
m\'as, que llamamos miscel\'anea, cuyo contenido puede variar de unos cursos a
otros. En esta parte usamos los rudimentos de programaci\'on que hemos visto
antes para resolver problemas concretos dentro de estas \'areas. 

Aunque puede parecer que no hay conexi\'on entre estos bloques, veremos que  la
hay bastante fuerte:
\begin{enumerate}
 \item La aritm\'etica, el estudio de los n\'umeros enteros, es la base de
muchos de los sistemas criptogr\'aficos que usamos para transmitir informaci\'on
de forma segura. En particular, estudiaremos el sistema RSA, uno de los m\'as
utilizados actualmente,  cuya seguridad se basa en la enorme dificultad de
factorizar un entero muy grande cuyos \'unicos factores son dos n\'umeros primos
enormes y distantes entre s\'{\i}. 
\item Para elegir los factores primos en el sistema RSA, cada usuario del
sistema debe tener su par de primos, se utilizan generadores de n\'umeros
(pseudo-)aleatorios.  Este ser\'a uno de los asuntos que trataremos en el bloque
de probabilidad. 
\item En el cap\'{\i}tulo de ampliaci\'on de teor\'{\i}a de n\'umeros discutiremos c\'omo encontrar n\'umeros primos muy grandes y c\'omo intentar factorizar, de manera eficiente, n\'umeros grandes. Son dos asuntos muy relacionados con la criptograf\'{\i}a.  
\item En el bloque sobre la aproximaci\'on de n\'umeros reales, dedicaremos
cierta atenci\'on al c\'alculo  de los d\'igitos de algunas constantes
matem\'aticas, en particular $\pi$ y $e$. Tambi\'en estudiaremos la resoluci\'on aproximada de ecuaciones y la aproximaci\'on de funciones. 
%%Nos encontraremos con una pregunta interesante: ?`se pueden considerar los%%
%%d\'igitos de $\pi$ como un generador de n\'umeros (pseudo-)aleatorios?%%

\item Como aplicaci\'on de este bloque sobre aproximaci\'on de n\'umeros reales
veremos un par de ejemplos  en que usaremos logaritmos para estudiar potencias
$a^n$ con $n$ muy grande. Es decir, usaremos los reales ({\itshape el continuo}) para estudiar un problema discreto. 
 \end{enumerate}


De la lista anterior se deduce que el tema central del curso es la
criptograf\'{\i}a,  con varios de los otros temas ayudando a entender y aplicar
correctamente los sistemas criptogr\'aficos.

Los temas que tratamos en la segunda parte del curso vuelven a aparecer en
asignaturas de la carrera, en alg\'un caso optativas como la criptograf\'{\i}a,
y pueden verse como peque\~nas introducciones {\itshape experimentales} a
ellos. Creemos entonces que la asignatura es una buena muestra, por supuesto
 incompleta,  de lo que os encontrar\'eis durante los pr\'oximos a\~os. Es como
catar {\itshape el mel\'on} antes de abrirlo.
 
 El prototipo de este curso fu\'e desarrollado, en su totalidad,  por Pablo
 Angulo, y puedes todav\'{\i}a consultar el excelente resultado de su trabajo en
 este  \href{http://verso.mat.uam.es/~pablo.angulo/doc/laboratorio/}{enlace}. Los que lo hemos impartido despu\'es aprendimos casi todo lo que sabemos en sus notas y, por supuesto, se lo agradecemos aqu\'{\i}. 
 
 
 
 
 
 
 \smallskip




\begin{comment}
{\sc Algunos consejos}


No deja de ser una perogrullada  afirmar que para superar un curso en el que
la programaci\'on juega un papel destacado hay que programar, por uno mismo,
sistem\'aticamente. Dicho de otra manera, {\itshape se aprende a programar
programando.}

M\'as concretamente,
\begin{enumerate}
 \item No suele servir {\itshape cortar y pegar}  c\'odigos escritos por
otros. Frecuentemente esa manera de trabajar produce errores  en el
c\'odigo resultante, en ocasiones dif\'{\i}ciles de arreglar.

\item Antes de poder programar un algoritmo es necesario {\itshape entenderlo
bien}. Uno se da cuenta de que no entiende bien el algoritmo cuando encuentra
dificultades para expresarlo como un programa. En tales casos suele ser de ayuda
ejecutar {\itshape a mano} el algoritmo para valores muy bajos de sus
par\'ametros. 

\item Es f\'acil convencerse de que se ha comprendido un c\'odigo, escrito por
otros, que funciona correctamente. Sin embargo, frecuentemente s\'olo se
consigue una buena comprensi\'on comparando en detalle nuestro c\'odigo,
correcto o no, con otras soluciones, por ejemplo las del  profesor. 

\item Siempre hay que intentar comprobar las soluciones que produce nuestro
c\'odigo. Para valores bajos de los par\'ametros del c\'odigo podemos intentar
 {\itshape calcular a mano} las soluciones para comparar, y, en general, es
bueno pensar siempre si las soluciones obtenidas se ajustan a lo que
podr\'{\i}amos esperar {\itshape a priori.}
 
 
 
\end{enumerate}




\smallskip

{\sc Un poco de historia}

El prototipo de este curso fu\'e desarrollado, en su totalidad,  por Pablo
Angulo, y puedes todav\'{\i}a consultar el excelente resultado de su trabajo en
este 
\href{http://verso.mat.uam.es/~pablo.angulo/doc/laboratorio/}{enlace}.
Esta
versi\'on primera del curso estaba pensada tanto
para los estudiantes del grado en Matem\'aticas como para los del doble grado
Matem\'aticas-Inform\'atica. 

En aquel curso tambi\'en impartieron docencia en la asignatura Patricio
Cifuentes y Daniel Ortega.  En cursos sucesivos se vio la necesidad  de
adaptar m\'as el curso a los estudiantes del Grado en Matem\'aticas, que en
principio no tienen que saber programar antes de comenzar el curso, mientras que
los estudiantes del doble grado Matem\'aticas-Inform\'atica ya saben todos
programar antes de empezar. 

Posteriormente se incorporaron Bernardo L\'opez, Rafael Hern\'andez, Fernando
Quir\'os, Juan Ram\'on Esteban, Enrique Gonz\'alez, Adolfo Quir\'os, . El estado
actual del curso para el grado en Matem\'aticas, el que reflejan estas notas, 
corresponde al original de Pablo
Angulo,  con algunos a\~nadidos y con la adaptaci\'on ya mencionada a la
situaci\'on previa, respecto a la programaci\'on,  de muchos de los estudiantes
de grado. 


La idea que tenemos es seguir ampliando estas notas con nuevos temas, de forma
que dispongamos de un exceso de material,  y haya que seleccionar cada curso la
parte que se cubre. 

\end{comment}
