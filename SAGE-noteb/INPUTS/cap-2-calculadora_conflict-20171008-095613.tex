
%%%21-08-2013 Publicar 24-25-26-CAVAN, y dos  PDFs con ayuda para los ejercicios de calculo  24-CAVAN etc%%%%%%%%

Usamos {\sage} como una ``calculadora avanzada'' escribiendo en una celda de
c\'alculo una instrucci\'on preprogramada junto con un n\'umero correcto de
par\'ametros.
 
Por ejemplo, existe una
instrucci\'on para factorizar n\'umeros enteros, \lstinline|factor|, que admite
como
par\'ametro un n\'umero entero,  de forma que debemos evaluar una celda que
contenga 
\lstinline|factor(2^137+1)| para calcular los factores primos del n\'umero
$2^{137}+1$.

Muchas instrucciones de SAGE tienen, adem\'as de par\'ametros obligatorios,
algunos par\'ametros opcionales. Obtenemos la informaci\'on sobre los
par\'ametros de una funci\'on escribiendo, en una celda de cálculo, el nombre de la función seguido de un 
paréntesis abierto y evaluando, o, más cómodo, clicando la tecla de tabulación.
Por ejemplo, tabulando tras escribir \lstinline|factor(| obtenemos la
informaci\'on 
sobre la funci\'on que factoriza enteros. 

Adem\'as, si s\'olo conocemos, o sospechamos,  algunas letras del nombre de la
funci\'on podemos escribirlas en la celda de c\'alculo y pulsar el tabulador, lo
que produce un desplegable con los nombres de todas las instrucciones que
comienzan por esas letras. En el desplegable podemos elegir la que nos parezca
m\'as pr\'oxima a lo que buscamos y vemos que se completan las letras que
hab\'{\i}amos escrito con el nombre completo que hemos elegido. 
Si necesitamos informaci\'on sobre los par\'ametros de la instrucci\'on podemos a\~nadir el
interrogante al final y evaluar.


Una instrucci\'on como \lstinline|factor(2^137+1)| decimos que es una funci\'on,
o
que est\'a en forma funcional, ya que se parece a una funci\'on matem\'atica que
se aplica a un n\'umero entero y produce sus factores.  Hay otra clase de
instrucciones a las que llamamos {\itshape m\'etodos} y que tienen una sintaxis
diferente 
\begin{center}
\verb|(objeto).metodo()|
\end{center}

Por ejemplo, tambi\'en podemos factorizar un entero mediante 
\begin{center}
\lstinline|(2^137+1).factor()|
\end{center}
\noindent donde $2^{137}+1$ es un objeto de clase {\itshape n\'umero entero} al
que
aplicamos el m\'etodo {\itshape factor.} Esta sintaxis procede de la
programaci\'on
{\itshape orientada a objetos} (OOP), y como Python es un lenguaje orientado a
objetos es natural que en SAGE aparezca esta sintaxis. 

Algunas instrucciones, como \lstinline|factor|, admiten las dos formas,
funciones y m\'etodos, pero otras s\'olo tienen uno de las dos. Veremos ejemplos
a lo largo del curso. 


Como SAGE tiene miles de instrucciones preprogramadas, para toda clase de tareas
en matem\'aticas, es muy \'util tener a mano un {\itshape chuletario} con las
instrucciones m\'as comunes\footnote{Estos {\itshape chuletarios} pertenecen a versiones anteriores de {\sage} y no han sido actualizados. Aunque la sintaxis de las instrucciones de {\sage} no var\'{\i}a mucho, en ocasiones hay peque\~nos cambios en la versi'on actual respecto a la informaci\'on que aparece en el chuletario.}.  Hay varios y aqu\'{\i} puedes encontrar enlaces a
algunos:
\label{chuletas}
\begin{enumerate}\itemsep=-2pt
 \item  \href{http://150.244.21.37/PDFs/CAVAN/VisitaSAGE.pdf}{En castellano.} 
 \item  \href{http://150.244.21.37/PDFs/CAVAN/quickref.pdf}{Preparada por el desarrollador
principal de SAGE William Stein (en ingl\'es).} 
 \item  \href{http://150.244.21.37/PDFs/CAVAN/quickref-calc.pdf}{C\'alculo.} 
 \item  \href{http://150.244.21.37/PDFs/CAVAN/quickref-linalg.pdf}{\'Algebra lineal.} 
 \item  \href{http://150.244.21.37/PDFs/CAVAN/quickref-nt.pdf}{Aritm\'etica.} 
 \item  \href{http://150.244.21.37/PDFs/CAVAN/quickref-algebra.pdf}{Estructuras algebraicas.} 
 \item  \href{http://150.244.21.37/PDFs/CAVAN/quickref-graphtheory.pdf}{Grafos.}
\end{enumerate}




\section{Aritmética elemental}

Las operaciones de \emph{la escuela} son sencillas de invocar. Sobre el
resultado a esperar de cada operación (o sucesión de operaciones), ha de tenerse
en cuenta que la aritmética es exacta, así:
\begin{itemize}
 \item Suma (y diferencia): \lstinline|3+56+23-75| devuelve \lstinline|7|.
 \item Producto: \lstinline|3*56*23|, \lstinline|3864|
 \item Potencia: \lstinline|3**2|, \lstinline|9|; y también lo hace
\lstinline|3^2|.\footnote{La segunda expresión, con ser más sencilla, obliga a
hacer aparecer el símbolo \^\ en escena, lo que, en la mayoría de teclados,
requiere pulsar la barra espaciadora tras él.}
 \item Cociente: \lstinline|3864/56| es \lstinline|23|. Pero, \lstinline|7/3| (o
\lstinline|14/6|) devolverá, \lstinline|7/3|.

También la respuesta a \lstinline|7/5| se muestra en notación
racional\footnote{El cuerpo de los racionales, $\mathbb Q$, es suficiente para
contener cocientes de números enteros.}, \lstinline|7/5|,  en detrimento de la
notación decimal\footnote{Se reserva la notación decimal para números que, con
cierta precisión (finita), servirán para aproximar reales, complejos, ...},
\lstinline|1.4|.

\item División entera: \lstinline|7//2| devuelve \lstinline|3|, el cociente en
la
división entera\footnote{\emph{Dividendo=cociente$\times$divisor$+$resto}, con 
\emph{$0<$resto$<$divisor}.};
y \lstinline|7%2|, el resto, \lstinline|1|. 
\end{itemize}

Al encontrar varias operaciones
en una misma expresión, se siguen las usuales reglas de precedencia, rotas por
la presencia de paréntesis.
\begin{itemize}
 \item \lstinline|8*(5-3)+9^(1/2)+6| da \lstinline|25|, mientras que
\lstinline|(8*(5-3)+9)^(1/2)+6| es \lstinline|11|.
\end{itemize}



Estas operaciones aritm\'eticas se efect\'uan en alg\'un conjunto, generalmente
un anillo o cuerpo, de n\'umeros,  y {\sage} dispone de  unos cuantos
predefinidos:

\begin{center}\label{Tabla-Ambientes}
\def\entrada#1 #2.{
\textcolor{blue}{#1} & #2 \\
\hline}
\begin{tabular}{|l|l|}
 \hline
 Símbolo & Descripción \\
 \hline
\entrada ZZ  anillo de los números enteros, $\mathbb{Z}$.
\entrada Integers(6)  anillo de enteros módulo $6$, $\mathbb Z_6$.
\entrada QQ cuerpo de los números racionales, $\mathbb Q$.
\entrada RR cuerpo de los números reales (53 bits de precisión), $\mathbb R$.
\entrada CC cuerpo de los números complejos (53 bits de precisión), $\mathbb C$.
\entrada RDF cuerpo de números reales con doble precisión.
\entrada CDF cuerpo de números complejos con doble precisión.
\entrada RealField(400) reales con 400-bit de precisión.
\entrada ComplexField(400) complejos con 400-bit de precisión.
\entrada ZZ[I] anillo de enteros Gaussianos.
\entrada {AA, QQbar} cuerpo de números algebraicos, Q.
\entrada FiniteField(7) cuerpo finito de 7 elementos, $\mathbb Z_7$ o $\mathbb
F_7$.
\entrada SR anillo de expresiones simbólicas.
\end{tabular}
\end{center}


\section{Listas y tuplas}

Las listas y tuplas son {\itshape estructuras de datos} y se tratar\'an m\'as
ampliamente en el cap\'{\i}tulo siguiente. Para cubrir las necesidades de este
cap\'{\i}tulo baste decir que
\begin{enumerate}
 \item Se crea una lista de nombre \lstinline|L| escribiendo en una celda de
c\'alculo algo como
\begin{lstlisting}
L=[1,7,2,5,27,-5]
\end{lstlisting}
 \item Los elementos de la lista est\'an ordenados  y se accede a ellos mediante
instrucciones como 
 \lstinline|a=L[1]|, que asigna a la variable $a$ el valor del segundo elemento
de la
lista, es decir $a$ vale $7$. Eso quiere decir que el primer elemento de la
lista es \lstinline|L[0]| y {\sc no} \lstinline|L[1]| como podr\'{\i}amos
esperar.
\item Las tuplas son muy parecidas a las listas, pero hay ciertas diferencias
que veremos m\'as adelante. Una tupla \lstinline|t| se crea con la
instrucci\'on 
\begin{lstlisting}
t=(1,7,2,5,27,-5)
\end{lstlisting}
 \noindent con par\'entesis en lugar de corchetes. Se accede a sus elementos de
la misma forma que para las listas, y, por ejemplo, 
\lstinline|t[1]| vale $7$. Mencionamos aqu\'{\i} las tuplas porque  en {\sage}
las
coordenadas de un punto no forman una lista sino una tupla.  
 
\end{enumerate}

\section{Funciones}

\label{variables}
Necesitamos definir funciones, en sentido matem\'atico, para poder calcular con
ellas,  o bien representarlas gr\'aficamente. Muchas funciones est\'an ya
definidas en {\sage} y lo \'unico que haremos es asignarles un nombre c\'omodo
para
poder referirnos a ellas. As\'i tenemos la exponencial, el logaritmo y las
funciones trigonom\'etricas.  Tambi\'en podemos construir nuevas funciones
usando las conocidas y las operaciones aritm\'eticas. 



Podemos definir una funci\'on, por ejemplo de la variable $x$, mediante una
expresi\'on como \verb|f(x)=sin(x)|, que asigna el nombre \verb|f| a la 
funci\'on
seno. A la derecha de la igualdad podemos escribir cualquier expresi\'on
definida usando las operaciones aritm\'eticas y las funciones definidas en
{\sage}.
De \'estas, las de uso m\'as com\'un son
\begin{center}
\def\entrada#1 #2.{
\textcolor{blue}{#1} & #2 \\
\hline}
\begin{tabular}{|l|l|}
 \hline
 Símbolo & Descripción \\
 \hline
\entrada $f1(x)=exp(x)$ que define la funci\'on exponencial, $e^x$.
\entrada $f2(x)=log(x)$ que define la funci\'on logaritmo neperiano, $\ln(x)$.
\entrada $f3(x)=sin(x)$ que define la funci\'on seno, $\mathrm{sen}(x)$.
\entrada $f4(x)=cos(x)$ que define la funci\'on coseno, $\cos(x)$.
\entrada $f5(x)=tan(x)$ que define la funci\'on tangente, $\tan(x)$.
\entrada $f6(x)=sqrt(x)$ que define la funci\'on ra\'{\i}z cuadrada, $\sqrt{x}$.
\end{tabular}
\end{center}
 

 Usando estas funciones y las operaciones aritm\'eticas podemos definir
funciones como 
 \begin{lstlisting}
  F(x,y,z)=sin(x^2+y^2+z^2)+sqrt(cos(x)+cos(y)+cos(z)).
 \end{lstlisting}
 
 Como definimos la función indicando expl\'{\i}citamente el orden de sus
argumentos no existe  ambig\"uedad sobre qu\'e variables se deben sustituir por
qu\'e argumentos:
\begin{lstlisting}
f(a,b,c) = a + 2*b + 3*c
f(1,2,1)
\end{lstlisting}
\begin{Output}
	8
\end{Output}
\pagebreak[3]
\begin{lstlisting}
s = a + 2*b + 3*c
s(1,2,1) 
\end{lstlisting}
\begin{Output}
	__main__:4: DeprecationWarning: Substitution using function-call syntax
	and unnamed arguments is deprecated and will be removed from a future
	release of Sage; you can use named arguments instead, like EXPR(x=...,
	y=...)
	See http://trac.sagemath.org/5930 for details.
	8
\end{Output}

En este ejemplo vemos que podemos sustituir en una funci\'on, pero la
sustituci\'on en una expresi\'on simb\'olica ser\'a imposible, dentro de
{\sage}, 
en un futuro
pr\'oximo.
 
 
Puedes encontrar m\'as informaci\'on acerca de las funciones predefinidas en
{\sage}  en esta
 \href{http://www.sagemath.org/doc/reference/functions/index.html}{p\'agina}.
 
 \medskip
 
 La otra manera de definir funciones matem\'aticas es usando la misma sintaxis
que utilizamos, en Python,  para definir programas o trozos de programas. Esto
aparecer\'a en detalle \hyperref[sect-funciones]{m\'as adelante}, pero de 
momento podemos ver un
ejemplo:
\begin{lstlisting}
 def f(x):
      return x*x
\end{lstlisting}
\noindent define la funci\'on {\itshape elevar al cuadrado.}






\subsection{Variables y expresiones simbólicas}

Una  \textbf{variable simbólica}  es un objeto en Python que representa una
{\itshape variable}, en sentido matem\'atico,  que
puede tomar cualquier valor en un cierto dominio. Casi siempre, ese dominio es
un cuerpo de n\'umeros, como los racionales, los reales o los n\'umeros
complejos. En el caso de los n\'umeros racionales la representaci\'on es exacta,
mientras que  los reales, o los complejos, se representan usando decimales con
un n\'umero dado de d\'{\i}gitos en la parte decimal.


\begin{enumerate}

\item La variable simbólica
\lstinline|x| está predefinida, y para cualquier otra que utilicemos
debemos
avisar al intérprete:
\begin{lstlisting}[numbers=none]
var('a b c')
\end{lstlisting}


\item La igualdad \lstinline|a = 2| es una asignaci\'on que crea
una variable $a$ y le da el valor $2$. Hasta que no se eval\'ue una celda que
le asigne otro valor, por ejemplo \lstinline|a = 3|, el valor de $a$
ser\'a $2$. Los valores de las variables, una vez asignados, se mantienen
dentro de la hoja mientras no se cambien expl\'{\i}citamente. 


\item Una operación que involucra una o más variables simbólicas no devuelve un
valor
numérico,
sino una \textbf{expresión simbólica} que involucra números, operaciones,
funciones y
variables simbólicas. 
\begin{lstlisting}[numbers=none]
s = a + b + c
s2 = a^2 + b^2 + c^2
s3 = a^3 + b^3 + c^3
s4 = a + b + c + 2*(a + b + c)
p = a*b*c
\end{lstlisting}
\item Una igualdad como \lstinline|s = a + b + c| es, de hecho, una
asignaci\'on: despu\'es de ejecutar esa l\'{\i}nea el valor de {\tt s} es
\lstinline|a + b + c| y, por tanto, el de $s^2$ es $(a+b+c)^2,$ etc. 


\item Podemos imprimirlas como código
\begin{lstlisting}[numbers=none]
print s; s2; s3; s4; p
\end{lstlisting}
\begin{Output}
	a + b + c
	a^2 + b^2 + c^2
	a^3 + b^3 + c^3
	3*a + 3*b + 3*c
	a*b*c
\end{Output}
 o mostrarlas en un formato matemático más habitual
\begin{lstlisting}[numbers=none]
show([s, s2, s3, s4, p])
\end{lstlisting}
\begin{Output}
	$\left[a + b + c, a^{2} + b^{2} + c^{2}, a^{3} + b^{3} + c^{3}, 3 \, a +
	3 \, b + 3 \, c, a b c\right]$
\end{Output}



\item Si en algún momento sustituimos las variables simbólicas por números
(o elementos de un anillo), podremos realizar las operaciones y obtener un
número (o un
elemento de un anillo).
\begin{lstlisting}[numbers=none]
print s(a=1, b=1, c=1), s(a=1, b=2, c=3)
s(a=1, b=1, c=1)+ s(a=1, b=2, c=3)
\end{lstlisting}
\begin{Output}
	3 6
	9
\end{Output}

La sustituci\'on no cambia el valor de $s$, que sigue siendo una expresi\'on 
simb\'olica, y s\'olo nos da el valor que se obtiene al sustituir. 
Obsérvese, en cambio, qué se obtiene si ejecutamos
\begin{lstlisting}[numbers=none]
a=1;b=1;c=1
print s
\end{lstlisting}

\item Si operamos expresiones simbólicas, obtenemos otras expresiones
simbólicas,
aunque pocas veces estarán simplificadas
\begin{lstlisting}[numbers=none]
ex = (1/6)*(s^3 - 3*s*s2 +2*s3 )
show(ex)
\end{lstlisting}
\begin{Output}
	$\frac{1}{6} \, {\left(a + b + c\right)}^{3} + \frac{1}{3} \, a^{3} +
	\frac{1}{3} \, b^{3} + \frac{1}{3} \, c^{3} - \frac{1}{2} \, {\left(a +
	b + c\right)} {\left(a^{2} + b^{2} + c^{2}\right)}$
\end{Output}
\end{enumerate}
\subsubsection{Simplificar expresiones}

Observamos que
al crear la expresión se han realizado ``\emph{de oficio}'' algunas
simplificaciones
triviales. En ejemplos como el de arriba, nos puede interesar simplificar la
expresión
todavía más, pero es necesario decir qué queremos exactamente.

\smallskip

Existen varias estrategias
para intentar simplificar una expresión, y cada estrategia puede tener más o
menos éxito
dependiendo del tipo de expresión simbólica. Algunas dan lugar a una expresión
más sencilla en algunos casos, pero no en otros, y con expresiones complicadas
pueden
consumir bastante tiempo de proceso.
\label{expand}
Para la expresión anterior, como tenemos un polinomio, es buena idea expandirla
en
monomios que se puedan comparar unos con otros, usando el método 
\lstinline|.expand()|.

\begin{lstlisting}[numbers=none]
show(ex.expand())
\end{lstlisting}
\begin{Output}
	$abc$
\end{Output}
\pagebreak[3]

A menudo nos interesa lo contrario: factorizar la expresión usando
\lstinline|.factor()|
\begin{lstlisting}[numbers=none]
p = a^3 + a^2*b + a^2*c + a*b^2 + a*c^2 + b^3 + b^2*c + b*c^2 + c^3
show(p)
show(p.factor())
\end{lstlisting}
\begin{Output}
	$a^{3}+a^{2}b+a^{2}c+ab^{2}+ac^{2}+b^{3}+b^{2}c+bc^{2}+c^{3}$
	${\left(a+b+c\right)}{\left(a^{2}+b^{2}+c^{2}\right)}$
\end{Output}

Si consultas con el tabulador los métodos de las expresiones simbólicas, verás
que hay
métodos específicos para expresiones con funciones trigonométricas,
exponenciales, con
radicales o fracciones (es decir, con funciones racionales), ...

\begin{minipage}{.3\textwidth}
\ilustra{metodos_simplify}
\end{minipage}


\begin{lstlisting}[numbers=none]
p = sin(3*a)
show(p.expand_trig())
\end{lstlisting}
\begin{Output}
	$-\sin(a)^3+3 \, \sin(a)\cos(a)^{2}$
\end{Output}

\begin{lstlisting}[numbers=none]
p = sin(a)^2 - cos(a)^2
show(p.simplify_trig())
\end{lstlisting}
\begin{Output}
	$-2\,\cos(a)^2+1$
\end{Output}

\begin{lstlisting}[numbers=none]
p = 2^a * 4^(2*a)
show(p.simplify_exp())
\end{lstlisting}
\begin{Output}
	$2^{5\,a}$
\end{Output}
\begin{lstlisting}[numbers=none]
p = 1/a - 1/(a+1)
show(p.simplify_rational())
\end{lstlisting}
\begin{Output}
	$\dfrac{1}{a^{2} + a}$
\end{Output}



\subsection{Variables booleanas}\label{bool}
Un tipo especial de variables es el \emph{booleano} o lógico. Una tal variable
toma
valores \verb|True| (verdadero) o \verb|False| (falso). \footnote{Aunque, como
veremos algo más adelante, también los valores $1$, en lugar de {\tt True},y $0$
en lugar de {\tt False}.}



El  doble signo igual ({\tt ==}) sirve para comparar,
y devuelve \lstinline|True| o \lstinline|False| según los objetos comparados
sean iguales o no para el intérprete. De la misma manera se pueden usar, siempre
que tengan sentido, las comparaciones \lstinline|a<b| y \lstinline|a<=b|. 
Como veremos en el cap\'{\i}tulo \ref{prog}, estas comparaciones aparecen en los
bucles \lstinline|while| y al bifurcar la ejecuci\'on de c\'odigo mediante un
\lstinline|if|.

\noindent\begin{minipage}{1\textwidth}
Operaciones básicas con variables booleanas son la \emph{conjunción}
(\lstinline|and|),
la \emph{disyunción} (\lstinline|or|) y la \emph{negación} (\lstinline|not|):

\mbox{}\hfill
\small
\begin{tabular}{*3{|l}|}
\hline
 \lstinline| and | & True & False\\
 \hline
 True & True & False\\
 \hline
 False & False & False\\
 \hline
\end{tabular}
\hfill
\begin{tabular}{*3{|l}|}
\hline
 \lstinline| or | & True & False\\
 \hline
 True & True & True\\
 \hline
 False & True & False\\
 \hline
\end{tabular}
\hfill
\begin{tabular}{*2{|l}|}
\hline
 & \lstinline| not | \\
 \hline
 True & False\\
 \hline
 False & True \\
 \hline
\end{tabular}
\hfill\mbox{}
\end{minipage}


\section{Gr\'aficas}




\subsection{Gr\'aficas en 2D}

Utilizaremos comandos como los que siguen para obtener objetos gráficos:

\begin{enumerate}\itemsep=-1pt
	\item  \lstinline|point(punto o lista)|, \lstinline|points(lista)|, %
	\lstinline|point2d(lista)|, \lstinline|point3d(lista)| %
	dibujan los puntos de la \lstinline|lista| que se pasa como
	argumento. Usaremos este tipo de gráficas, en particular,  al estudiar la
	\hyperref[bertrand]{paradoja de Bertrand} en el cap\'{\i}tulo \ref{prob}. 
	
	\item  \lstinline|line(lista)|, \lstinline|line2d(lista)|, %
	\lstinline|line3d(lista)| dibujan líneas entre los puntos de la lista que se
	pasa como
	argumento. Usaremos este tipo de gráficas, en particular,  al estudiar el
	problema conocido como \hyperref[]{la ruina del jugador} en el cap\'{\i}tulo
	\ref{prob}.
	
	\item \lstinline|plot(f,x,xmin=a,xmax=b)|  esboza una gráfica de la función de una
	variable  %
	\lstinline|f| en el intervalo $[a,b]$. La escala en el eje $OY$ la elige {\sage} autom\'aticamente. En ocasiones, debido a esa elecci\'on autom\'atica,  la gr\'afica mostrada consiste en un trozo encima del eje $OX$ y otro pr\'acticamente vertical, lo que no resulta muy \'util. Podemos centrarnos en la parte que nos interesa del plano mediante 
	\lstinline|plot(f,x,xmin=a,xmax=b,ymin=c,ymax=d)|, que muestra la parte de la gr\'afica contenida en el rect\'angulo $[a,b]\times [c,d].$
	
		\item \lstinline|parametric_plot([f(t),g(t)],(t,0,2))|  esboza una gráfica de la
	curva dada en param\'etricas mediante  dos funciones de $t$,    %
	\lstinline|f|~y~\lstinline|g|,  con la variable $t$ variando en el intervalo
	$[0,2]$.
	
	\item Mediante \lstinline|implicit_plot(f,(x,-5,5),(y,-5,5))| obtenemos una
	representaci\'on de la parte de  la curva $f(x,y)=0$ contenida en el
	cuadrado $[-5,5]\times [-5,5]$. Decimos que se trata de una representaci\'on de
	una curva {\itshape ``en impl\'{\i}citas''.}
	
\end{enumerate}

Puedes ver algunos ejemplos de gr\'aficas en el  plano  en la hoja de {\sage}
\href{http://localhost:8888/notebooks/CAVAN/21-CAVAN-graficas.ipynb}{\tt 21-CAVAN-graficas.ipynb}.

\subsection{Gr\'aficas en 3D}

El sistema para representaciones gr\'aficas en 3D que tiene {\sage} no funciona en {\tt jupyterhub}, es decir no muestra los gr\'aficos cuando se ejecuta en el servidor del Departamento. En cambio, desde la versi\'on {\bf 8.0} de {\sage},  funciona bien cuando se ejecuta en la versi\'on local de {\tt Jupyter}. Entonces, y hasta que se resuelva este problema, en el servidor  usaremos {\tt matplotlib}.

\begin{enumerate}
\item En general, {\tt matplotlib} produce gr\'aficos 3D a partir de listas de coordenadas de puntos, de la forma $(x,y,f(x,y))$,  en $\mathbb{R}^3$ . As\'{\i} por ejemplo, para representar la gr\'afica de una funci\'on de dos variables  $f(x,y)$ primero necesitamos una lista con coordenadas de puntos en el rect\'angulo $[a,b]\times[c,d]$ sobre el que deseamos obtener la gr\'afica. 

\begin{lstlisting}[numbers=none]
x = np.arange(-2,2,0.05)
y = np.arange(-2,2,0.05)
X,Y = np.meshgrid(x, y)
\end{lstlisting}	

Esto nos dar\'a $80=4/0.05$ puntos en el intervalo $[-2,2]$ para la variable $x$, otros $80$ para la $y$, y  un ret\'{\i}culo de $6400$ puntos, uniformemente distribuidos,  en el  cuadrado $[-2,2]\times [2,2]$. Los comandos que empiezan por $\tt np$ son de {\tt numpy}, y permiten la manipulaci\'on eficiente de listas y matrices. Veremos m\'as sobre esto m\'as adelante.

\item Ahora definimos la funci\'on que vamos a representar
\begin{lstlisting}[numbers=none]
def f(x,y):
	return np.cos(x**2+y**2)
\end{lstlisting}	
	
Como usamos listas y matrices de {\tt numpy}, las funciones matem\'aticas como el coseno tambi\'en deben ser de {\tt numpy}, y por eso usamos {\tt np.cos.}
\item Finalmente, representamos la superficie mediante, por ejemplo, 
\begin{lstlisting}[numbers=left]
fig = plt.figure()
ax = fig.add_subplot(111, projection='3d')
zs = np.array([f(x,y) for x,y in zip(np.ravel(X), np.ravel(Y))])
Z = zs.reshape(X.shape)
ax.plot_surface(X,Y,Z, cmap=plt.cm.rainbow,rstride=1, cstride=1)
\end{lstlisting}	


La tercera l\'{\i}nea es la que crea las coordenadas de los puntos en $\mathbb{R}^3$ que usamos para construir la superficie, y la quinta es la que finalmente genera el gr\'afico interactivo.



\item Para obtener el gr\'afico primero, es decir en la primera celda que ejecutamos,  hay que importar una serie de paquetes 

\begin{lstlisting}
import numpy as np
import matplotlib as mpl
import matplotlib.pyplot as plt
from mpl_toolkits.mplot3d import Axes3D
from math import *
%matplotlib notebook
\end{lstlisting}


\end{enumerate}


\ilustra{superficie}





Puedes ver un resumen del uso de {\tt matplotlib} en \href{http://pausa.mat.uam.es/PDFs/CAVAN/matplotlib.pdf}{este tutorial}\footnote{{\tt IPython} es el nombre antiguo de {\tt Jupyter}, pero la mayor parte de lo que se dice en este tutorial funciona en {\tt Jupyter} sin problema.} y algunos ejemplos en la hoja 
\begin{center}
 \href{http://localhost:8888/notebooks/CAVAN/21-CAVAN-graficas-matplotlib.ipynb}{\tt 21-CAVAN-graficas-matplotlib.ipynb}.
\end{center}

\subsection{Gr\'aficas 3D en el {\tt notebook} antiguo}
Las tres funciones principales para representaciones gr\'aficas 3D son

\begin{enumerate}
	
	\item \lstinline|plot3d(g,(x,-10,10),(y,-10,10))|  esboza una gráfica de la
	función de dos variables  %
	\lstinline|g| sobre el cuadrado  $[-10,10]\times [-10,10]$.
	
	
	\item \lstinline|parametric_plot3d([f(u,v),g(u,v),h(u,v)],(u,0,2),(v,0,2))| 
	esboza una gráfica de la
	superficie  dada en param\'etricas mediante  tres  funciones de $u$ y $v$,    %
	\lstinline|f|, \lstinline|g| y \lstinline|h|,  con las variables $u$ y $v$  en 
	el intervalo $[0,2]$.
	
	Los puntos de la superficie son entonces los que se obtienen mediante
	$x=f(u,v),y=g(u,v),z=h(u,v).$
	
	\item Con \lstinline|implicit_plot3d(f,(x,-5,5),(y,-5,5),(z,-5,5))| 
	se representa una parte de  la superficie $f(x,y,z)=0$ contenida en el
	cubo $[-5,5]\times [-5,5]\times [-5,5]$. Decimos que se trata de una
	representaci\'on de una superficie {\itshape ``en impl\'{\i}citas''.}
	\end{enumerate}

Si se ejecutan en {\tt Jupyter} se obtiene un gr\'afico vac\'{\i}o, pero en el {\tt notebook} antiguo de {\sage} (\ref{ini-ses}) funcionan perfectamente. 

\section{C\'alculo}


En esta secci\'on usaremos la hoja de {\sage} 
\href{http://localhost:8888/notebooks/CAVAN/22-CAVAN-calculo-s1.ipynb}{\tt 22-CAVAN-calculo-s1.ipynb}.

\medskip

Cuando usamos el ordenador para estudiar problemas de c\'alculo diferencial o
integral operamos frecuentemente con valores aproximados de los n\'umeros reales
implicados, es decir truncamos los n\'umeros reales despu\'es de un n\'umero
prefijado de decimales. 

Sin embargo, en {\sage} podemos realizar gran cantidad de operaciones de manera
``simb\'olica'', es decir,  sin evaluar de manera aproximada los n\'umeros
reales o complejos implicados. As\'{\i}, por ejemplo, \lstinline|sqrt(2)| es una
representaci\'on simb\'olica de $\sqrt{2}$, una expresi\'on cuyo cuadrado es
exactamente $2$,  mientras que  \lstinline|sqrt(2).n()| (o bien
\lstinline|n(sqrt(2))|) es un valor aproximado con $20$ decimales y su cuadrado
no es exactamente $2$.
\label{num}
Para obtener una
representación decimal de una expresión simbólica, podemos usar los comandos
\lstinline|n()|, \lstinline|N()|,  o los métodos homónimos \lstinline|.n()|,
\lstinline|.N()| (n de numérico).\footnote{Es muy
habitual la asignación `$n=\ $', o `$N=\,$', cuando pensamos en dar nombre a una
variable
entera. Si la utilizamos en una hoja de {\sage}, el sistema no nos avisa de que
estamos
\emph{tapando las funciones} \lstinline|n(), N()|. Desde ese momento, y hasta
que no
reiniciemos la hoja, ya no funcionará la función. Se recomienda el uso de los
métodos,
\lstinline|.n(), .N()|, para evitar innecesarios dolores de cabeza.}


El c\'alculo con valores aproximados de n\'umeros reales necesariamente
introduce errores, y suele llamarse {\itshape C\'alculo num\'erico} a la materia
que estudia c\'omo controlar esos errores, de manera que conocemos el grado de
validez del resultado final y tratamos de que sea lo m\'as alto posible. 
Volveremos en el cap\'{\i}tulo \ref{aprox} sobre este asunto. 

La diferencia entre el {\itshape c\'alculo simb\'olico} y el {\itshape c\'alculo
num\'erico} es importante: el c\'alculo simb\'olico es exacto pero m\'as
limitado que el num\'erico. Un ejemplo t\'{\i}pico podr\'{\i}a ser el c\'alculo
de una integral definida 
\begin{equation}\label{int1}
\int_a^b f(x)\ dx,
\end{equation}
\noindent que simb\'olicamente requiere el c\'alculo de una primitiva $F(x)$ de
$f(x)$ de forma que el valor de la integral definida es $F(b)-F(a)$. El
c\'alculo de primitivas no es f\'acil, y para algunas funciones, como por
ejemplo $\sin(x)/x$, la primitiva no se puede expresar usando las 
funciones
habituales y, si fuera necesario, deber\'{\i}amos considerarla como una nueva
funci\'on elemental semejante a las trigonom\'etricas o la exponencial.

En cambio, para cada funci\'on continua $f(x)$ en un intervalo $[a,b]$ podemos 
calcular f\'acilmente un valor aproximado para la integral definida 
(\ref{int1}). Veremos alguno de estos m\'etodos en la secci\'on  \ref{monte}.


\subsection{Ecuaciones}
\begin{enumerate}
  \item El comando  \lstinline|solve()|  permite resolver ecuaciones (al menos
lo
intenta): para resolver una ecuación llamamos a \lstinline|solve()| con la
ecuación como primer
argumento y la variable a despejar como segundo argumento, y recibimos una lista
con las
soluciones.
\begin{lstlisting}[numbers=none]
#Las soluciones de esta ecuaci$\X o$n c$\X u$bica son n$\X u$meros complejos
solve(x^3 - 3*x + 5, x)
\end{lstlisting}

 \item El algoritmo resuelve también un sistema de ecuaciones. Basta pasar la
\emph{lista} de igualdades o expresiones.
 \begin{lstlisting}[numbers=none]
var('x y')
solve([x^2+y^2==1,x-y+1],x,y)
\end{lstlisting}
\item El comando \lstinline|solve()| intenta obtener soluciones exactas en forma
de expresi\'on simb\'olica,  de la ecuaci\'on o sistema,
y frecuentemente no encuentra ninguna. Tambi\'en es posible buscar soluciones
aproximadas mediante m\'etodos num\'ericos:  el comando
\lstinline|find_root(f,a,b)| busca una soluci\'on de la ecuaci\'on $f(x)=0$ 
perteneciente al intervalo $[a,b]$. Volveremos sobre este asunto en la secci\'on
\ref{raices}.
 
\end{enumerate}


\subsection{L\'{\i}mites}
\begin{enumerate}
 \item El método  \lstinline|.limit()| (o la función \lstinline|limit()|) 
permite calcular
límites de funciones. Para calcular el límite de \verb|f| en un punto:
\begin{lstlisting}[numbers=none]
f1=x^2
print f1.limit(x=1)
\end{lstlisting}
 \item Tambi\'en se puede calcular el límite cuando la variable tiende a
infinito

\begin{lstlisting}[numbers=none]
f2=(x+1)*sin(x)
f3=f2/f1
f3.limit(x=+oo)
\end{lstlisting}
\item Si Sage sabe que el límite no existe, por ejemplo para la funci\'on
$f(x)=1/(x*sin(x))$, la función \lstinline|limit()| devuelve el
valor \lstinline|und|, abreviatura de \emph{undefined},  o el valor
\lstinline|ind|, que indica que no existe l\'{\i}mite pero
la funci\'on permanece acotada, por ejemplo $\sin(x)$ cuando $x$ tiende a
infinito,   cerca del valor l\'{\i}mite de la variable.

Por \'ultimo, en algunos casos Sage no sabe c\'omo determinar el l\'{\i}mite o
no sabe que no existe, y entonces devuelve la definici\'on de la funci\'on. Esto
ocurre, por ejemplo, con la funci\'on $f(x):=x/\sin(1/x)$ cuando se quiere
calcular el l\'{\i}mite cuando $x$ tiende a cero.

 \item También podemos calcular así límites de sucesiones. Al fin y al cabo, si
la
expresión simbólica admite valores reales, el límite de la función que define en
infinito, si existe, es el mismo que el límite de la sucesión de naturales
definido por la misma expresión.
\end{enumerate}




\subsection{Series}

Una serie es un tipo particular de sucesi\'on: dada una sucesi\'on
$a_0,a_1,a_2,\dots$  queremos dar sentido a la suma infinita 
\[S=a_0+a_1+a_2+\dots+a_n+\dots\]

Definimos $S$ como el l\'{\i}mite, si existe, de la sucesi\'on de ``sumas
parciales'' $S_n:=a_0+a_1+a_2+\dots+a_n$ cuando $n$ tiende a infinito y decimos
que $S$ es la suma de la serie. 

Para que pueda existir un l\'{\i}mite para $S_n$ tiene que ocurrir que el
l\'{\i}mite de $a_n$, cuando $n$ tiende a infinito, sea cero. Es una condici\'on
necesaria pero {\sc no suficiente} para la existencia de suma de la serie. Por
ejemplo, la sucesi\'on $a_n:=1/n$ tiende a cero cuando $n$ tiende a infinito,
pero  el l\'{\i}mite de la correspondiente sucesi\'on de sumas parciales es
infinito. 


El comando \lstinline|sum()|, tomando $\infty$ como l\'{\i}mite superior,
permite, a veces, 
calcular la suma de una serie infinita:
\begin{lstlisting}[columns=fullflexible,backgroundcolor=\color{white}]
sum(expresion, variable, limite_inferior, limite_superior)
\end{lstlisting}

\begin{lstlisting}
sum(1/k^2, k, 1, oo).show()
\end{lstlisting}
\begin{Output}
	$\frac16\pi^2$
\end{Output}

En casos como el anterior, Sage no realiza ning\'un c\'alculo para devolvernos
su respuesta. Lo \'unico que hace es identificar la serie que queremos sumar y
nos devuelve el valor de la suma si lo conoce. Por ejemplo, si le pedimos la
suma de los inversos de los cubos de enteros nos dice que vale
\lstinline|zeta(3)|, que no es sino el nombre que se usa en Matem\'aticas para
esa suma. Estudiaremos en detalle el c\'alculo con series en el cap\'{\i}tulo
\ref{aprox}. 









\subsection{C\'alculo diferencial}
\begin{enumerate}
 \item El método  \lstinline|.derivative()|, o la función 
\lstinline|derivative()|, permite
calcular derivadas de funciones simbólicas. Las \textbf{derivadas}  se obtienen
siguiendo
metódicamente las reglas de derivación y no suponen ningún problema al
ordenador:
\begin{lstlisting}[numbers=none]
f=1/(x*sin(x))
g=f.derivative()
show([g,g.simplify_trig()])
\end{lstlisting}


\item Se pueden calcular derivadas de órdenes superiores
\begin{lstlisting}[numbers=none]
m=5
[derivative(x^m,x,j) for j in range(m+1)]
\end{lstlisting}

Para derivar funciones de varias variables, es decir calcular derivadas
parciales (i.e. derivar una funci\'on de varias variables respecto a una de
ellas manteniendo las otras variables en valores constantes), basta especificar
la variable con respecto a la que derivamos y
el orden hasta el que derivamos: 
\begin{lstlisting}[numbers=none]
F(x,y)=x^2*sin(y)+y^2*cos(x)
show([F.derivative(y,2),derivative(F,x,1).derivative(y,1)])
\end{lstlisting}
\end{enumerate}

\subsection{Desarrollo de Taylor}\label{taylor}
As\'{\i} como la derivada  de una funci\'on en un punto $x_0$ nos permite
escribir una aproximaci\'on lineal de la funci\'on {\sc cerca del punto}
\[f(x)\sim f(x_0)+f^{\prime}(x_0)(x-x_0),\]
\noindent el polinomio de Taylor nos da una aproximaci\'on polinomial, de un
cierto grado prefijado, v\'alida {\sc cerca del punto}. As\'{\i} por ejemplo:
\begin{lstlisting}[numbers=none]
f(x)=exp(x)
taylor(f,x,0,20)
\end{lstlisting}
\noindent nos devuelve un polinomio de grado $20$ que cerca del origen aproxima 
la funci\'on exponencial. Volveremos sobre este \hyperref[taylor]{asunto} en el
cap\'{\i}tulo
\ref{aprox}.

\subsection{C\'alculo integral}
El cálculo de {\itshape primitivas}  es mucho  más complicado 
que la derivación, ya que no existe ningún método que pueda calcular
la primitiva de cualquier función. En realidad, hay muchas funciones elementales
(construídas
a partir
de funciones trigonométricas, exponenciales y algebraicas mediante sumas,
productos y
composición de funciones) cuyas primitivas, aunque estén bien definidas, no se
pueden
expresar en términos de estas mismas funciones. Los ejemplos $f(x)=e^{-x^2}$ y
$f(x)=\frac{\sin(x)}{x}$ son bien conocidos.

\

Aunque en teoría existe un algoritmo (el algoritmo de
\href{http://en.wikipedia.org/wiki/Risch\_algorithm}{Risch}) capaz de decidir si
la
primitiva de una función elemental es elemental, dificultades
prácticas
imposibilitan llevarlo a la práctica, y el resultado es que incluso en casos en
los que
integramos una función cuya primitiva es una función elemental, nuestro
algoritmo
de integración simbólica puede no darse cuenta. Si {\sage} no puede calcular una
primitiva de $f(x)$
expl\'{\i}cita devuelve  
\[\int f(x)\ dx.\]

\begin{enumerate}
 \item Los métodos (funciones) para el cálculo de primitivas a nuestra
disposición son:
%
\lstinline|.integral()| e \lstinline|.integrate()|.
\begin{lstlisting}[numbers=none]
f=1/sin(x)
show(f.integrate(x))
\end{lstlisting}
 \item Siempre podemos obtener una aproximación
numérica de una integral definida: 
\begin{lstlisting}[numbers=none]
f=tan(x)/x
[numerical_integral(f,pi/5,pi/4),N(f.integrate(x,pi/5,pi/4))]
\end{lstlisting}

Obsérvese que, con \lstinline|numerical_integral()|, el intérprete devuelve una
tupla, con
el valor aproximado de la integral y una cota para el error.
 
 \item Tambi\'en es posible
\href{
http://docs.scipy.org/doc/scipy/reference/tutorial/integrate.html#general-multip
le-integration-dblquad-tplquad-nquad}{integrar num\'ericamente funciones de
varias variables}, aunque,  de momento,  no vamos a tratar el tema. As\'{\i}
como las integrales de funciones de una variable corresponden al c\'alculo de
\'areas, las integrales de funciones de varias variables permiten calcular
vol\'umenes e hipervol\'umenes. 
 
 
 En el cap\'{\i}tulo \ref{prob}
veremos un m\'etodo, conocido como \hyperref[int-mc]{integraci\'on de Monte
Carlo}
que permite calcular valores aproximados de integrales de funciones de una o
varias variables. 
 
 
 
 
 
 \end{enumerate}


\section{\'Algebra lineal}



Podemos decir que el \'Algebra Lineal, al menos en espacios de dimensi\'on
finita,  trata de la {\itshape resoluci\'on de todos aquellos problemas que
pueden reducirse a encontrar las soluciones de un sistema de ecuaciones de
primer grado en todas las inc\'ognitas,} es decir, un sistema lineal de
ecuaciones.  Es posible realizar todas las operaciones necesarias para resolver
tales sistemas mediante c\'alculo con matrices, y esa es la forma preferida para
resolver problemas de \'Algebra Lineal mediante ordenador.

Los sistemas de  ecuaciones lineales con coeficientes en un cuerpo, por ejemplo
el de los n\'umeros racionales, siempre se pueden resolver y encontrar
expl\'{\i}citamente todas las soluciones del sistema en ese cuerpo o en uno que
lo contenga.  Esto no es cierto para ecuaciones polinomiales de grado m\'as
alto, y, por ejemplo, no hay m\'etodos generales para resolver una \'unica
ecuaci\'on  polinomial, de grado arbitrario $n$,  en una \'unica variable
\[a_0+a_1x+a_2x^2+\dots+a_nx^n=0.\]

Veremos en \hyperref[grobner]{la parte final del curso} algo de los que se puede
decir acerca de la resoluci\'on de los sistemas de ecuaciones polinomiales. 

\subsection{Construcción de matrices}
El constructor básico de matrices en Sage es la función \lstinline|matrix()|,
que,  en su forma m\'as simple, tiene como argumentos 
\begin{enumerate}
 \item El conjunto de n\'umeros, enteros, racionales, etc.,  al que pertenecen
las entradas de la matriz. 
 \item Una lista cuyos elementos son listas de la misma longitud,  y que ser\'an
las {\itshape filas} de la matriz.
\end{enumerate}
\begin{lstlisting}
A=matrix(ZZ,[[1,2,3],[4,5,6]]);A;show(A)
\end{lstlisting}

Una variante \'util contruye la misma matriz con 
\begin{lstlisting}
A=matrix(ZZ,2,[1,2,3,4,5,6]);show(A)
\end{lstlisting}
\noindent que trocea la lista dada como tercer argumento en el n\'umero de filas
dado por el segundo, en este caso $2$ filas. Por supuesto,  el n\'umero de
elementos de la lista debe ser m\'ultiplo del segundo argumento o aparece un
error. 

Para más ejemplos e información sobre \lstinline|matrix()|, pulsar el 
\verb|tabulador| del teclado tras escribir \lstinline|matrix(| en una celda.


\subsection{Submatrices}

Sobre el acceso a las entradas de una matriz se ha de tener en cuenta que tanto
filas como columnas empiezan su numeración en $0$. Así, en la matriz $3\times2$
definida por \lstinline|A=matrix(3,range(6))|, es decir
$$
\Biggl(
\begin{array}{*2{c}}
 0 & 1 \\ 2 & 3 \\ 4 & 5
\end{array}
\Biggr)
$$
el extremo superior izquierdo tiene \emph{índices} `0,0', y el inferior derecho,
`2,1'. Una vez claros los índices de un elemento, Sage usa
la notación \emph{slice} (por cortes) de Python para acceder a él:
\lstinline|A[0][0]| y \lstinline|A[2][1]| serían los elementos recién
mencionados. Para el caso especial de matrices, se ha adaptado también una
notación más habitual: \lstinline|A[0,0]| y \lstinline|A[2,1]|. 

En la notación \emph{slice}, dos índices separados por dos puntos, %
\lstinline|i:j|, indican el corte entre el primer índice, incluido,
hasta el segundo, que se excluye. En el ejemplo, se muestran las filas de
índices $2$ y $3$ de la matriz. 
Si se excluye alguno de los índices, se toma, por defecto, el valor más
extremo. 

\begin{lstlisting}
A=matrix(ZZ,10,[1..100])
A[7:]
\end{lstlisting}
\begin{Output}
	[ 71  72  73  74  75  76  77  78  79  80]
	[ 81  82  83  84  85  86  87  88  89  90]
	[ 91  92  93  94  95  96  97  98  99 100]
\end{Output}
La matriz \lstinline|A[7:]| consiste en las filas desde la octava, que tiene
\'{\i}ndice $7$ porque hemos empezado a contar en $0$, hasta la d\'ecima.

\begin{lstlisting}
A=matrix(10,[1..100])
A[:2]
\end{lstlisting}
\begin{Output}
 	[ 1  2  3  4  5  6  7  8  9 10]
	[11 12 13 14 15 16 17 18 19 20] 
\end{Output}
La matriz \lstinline|A[:2]| consiste en las filas primera y segunda, es decir,
las de \'{\i}ndice menor estrictamente que $2$.
Utilizando el doble índice, podemos recortar recuadros más concretos:

\begin{lstlisting}
A=matrix(10,[1..100])
A[3:5,4:7]
\end{lstlisting}
\begin{Output}
	[35 36 37]
	[45 46 47]
\end{Output}

Unas últimas virguerías  gracias a la notación slice: saltos e índices
negativos. La matriz $A$ es, como en el ejemplo anterior, la matriz $10\times
10$ con los primeros cien enteros colocados en orden en las filas de $A$. 
\begin{lstlisting}
A[2:5:2,4:7:2]
\end{lstlisting}

\noindent produce 

\begin{Output}
	[25  27]
	[45  47]
\end{Output}
\noindent se queda con las filas con \'{\i}ndice del $2$ al $4$,
saltando de dos en dos debido al $2$ que aparece en tercer lugar en
\lstinline|2:5:2|, y con las columnas con \'indices del $4$ al $6$
tambi\'en saltando de dos en dos por el que aparece en \lstinline|4:7:2|.
Siempre hay que recordar que las filas y
columnas se numeran empezando en el cero.  

Por otra parte, tambi\'en podemos usar \'{\i}ndices negativos lo que conduce a
quedarnos con filas o columnas que contamos hacia atr\'as, de derecha a
izquierda,  empezando por las
\'ultimas:

\begin{lstlisting}
A[-1];A.column(-2)
\end{lstlisting}
\begin{Output}
	(91, 92, 93, 94, 95, 96, 97, 98, 99, 100)
	(9, 19, 29, 39, 49, 59, 69, 79, 89, 99)
\end{Output}

La primera fila del resultado, que corresponde a \lstinline|A[-1]|, es la
\'ultima fila de la matriz, y la segunda es la novena columna (que tiene
\'{\i}ndice $8$). La \'ultima columna se obtendr\'{\i}a con
\lstinline|A.column(-1)|.
	
	
\subsection{Operaciones con matrices}


Las operaciones entre matrices, suma, diferencia, multiplicaci\'on, potencia e
inversa, se denotan con los mismos s\'{\i}mbolos que las correspondientes
operaciones entre n\'umeros.  En el caso de operaciones entre matrices es
posible que la operaci\'on~no sea posible porque los tama\~nos de las matrices
implicadas no sean compatibles, o en el caso de la inversa porque la matriz 
no~sea cuadrada de rango m\'aximo. 




De entre todas las funciones y m\'etodos que se aplican a matrices hemos
seleccionado las que parecen m\'as \'utiles: 

\def\bfitem#1:{\item\lstinline|#1|.- }
\begin{itemize}
\renewcommand{\labelitemi}{$\circ$}
 \bfitem identity_matrix(n): matriz identidad $n\times n$  
 \bfitem A.det(), det(A): determinante de $A$
 \bfitem A.rank(), rank(A): rango de $A$
 \bfitem A.trace(): traza de $A$
 \bfitem A.inverse(): inversa de $A$
 \bfitem A.transpose(): traspuesta de $A$
 \bfitem A.adjoint(): matriz adjunta de $A$
 \bfitem A.echelonize(), A.echelon_form(): matriz escalonada de $A$, en el
menor anillo en que vivan sus entradas
\bfitem A.rref(): matriz escalonada de $A$, en el menor cuerpo en que coincidan
sus entradas
\end{itemize}

Dado que resolver un sistema lineal de ecuaciones consiste, desde un punto de
vista matricial, en obtener la forma escalonada de la matriz (reducci\'on
gaussiana), la instrucci\'on quiz\'a m\'as importante en la lista es
\lstinline|A.echelon_form().|

Aunque \lstinline|A.echelonize()| y \lstinline|A.echelon_form()| hacen
esencialmente lo mismo, la primera deja la forma escalonada en la variable $A$,
por tanto machaca el valor antiguo de $A$,  y no es posible asignar el resultado
a otra variable, es decir \lstinline|B = A.echelonize()| no funciona, mientras
que  la segunda forma funciona como esperamos \lstinline|B=A.echelon_form()|
hace que 
la forma escalonada quede en $B$ y la matriz $A$ todav\'{\i}a existe con su
valor original. Puedes comprobar este comportamiento en la hoja 
\href{http://localhost:8888/notebooks/CAVAN/23-CAVAN-reduccion-gaussiana.ipynb}{\tt
23-CAVAN-reduccion-gaussiana.ipynb}.

\subsection{Espacios vectoriales}

Operar con matrices en el ordenador {\itshape siempre} ha sido posible debido a
que los lenguajes de programaci\'on disponen de la estructura de datos
\lstinline|array|, que esencialmente es lo mismo que una matriz. En los sistemas
de c\'alculo algebraico, como {\sage}, existe la posibilidad de definir objetos
matem\'aticos mucho m\'as abstractos como, por ejemplo,
\href{http://150.244.21.37/PDFs/CAVAN/quickref-algebra.pdf}{grupos},  
\href{http://150.244.21.37/PDFs/CAVAN/quickref-linalg.pdf}{espacios vectoriales}, 
\href{http://150.244.21.37/PDFs/CAVAN/quickref-graphtheory.pdf}{grafos}, 
\href{http://150.244.21.37/PDFs/CAVAN/quickref-algebra.pdf}{anillos}, etc. 




Nos fijamos en esta subseccci\'on  en el caso de los espacios vectoriales.
?`Qu\'e significa definir, como objeto de {\sage},  un espacio vectorial $E$ de
dimensi\'on $3$ sobre el cuerpo de los n\'umeros racionales?

Lo definimos mediante 
\lstinline|E = VectorSpace(QQ,3)| y sus elementos, vectores, mediante, por
ejemplo, \lstinline|v = vector([1,2,3])| o bien \lstinline|v = E([1,2,3])|.

Una vez definido el espacio vectorial y vectores en \'el, podemos realizar
c\'alculos con vectores siguiendo las reglas que definen, en  matem\'aticas, los
espacios vectoriales. As\'{i} por ejemplo, el sistema sabe que
$\mathbf{v}+(-1)*\mathbf{v}=\mathbf{0}.$

Tambi\'en es posible construir espacios  de matrices con una instrucci\'on como 
\lstinline|M = MatrixSpace(QQ, 4, 4)|, que define $M$ como el conjunto de
matrices con coeficientes racionales $4\times 4$,  con operaciones de suma,
producto por escalares, producto de matrices y matriz inversa.








\subsection{Subespacios vectoriales}\label{b3s3:index-2}
\begin{enumerate}
\item Podemos definir fácilmente el subespacio engendrado por un conjunto de
vectores, y después realizar operaciones como intersección o suma de
subespacios, o
comprobaciones como igualdad o inclusión de subespacios. 
\begin{lstlisting}
V1 = VectorSpace(QQ,3)
v1 = V1([1,1,1])
v2 = V1([1,1,0])
L1 = V1.subspace([v1,v2])
print L1
\end{lstlisting}


\item Definido un subespacio, podemos averiguar información sobre él:
\begin{lstlisting}
print dim(L1); L1.degree(); L1.ambient_vector_space()
\end{lstlisting}

{\sage} llama {\itshape degree} de un subespacio a la dimensi\'on de su espacio
ambiente.

\item Muchos operadores actúan sobre subespacios vectoriales, con los
significados
habituales.
\begin{lstlisting}
# Pertenencia a un subespacio
v3, v4 = vector([1,0,1]), vector([4,4,3])
print v3 in L1; v4 in L1
\end{lstlisting}

\begin{lstlisting}
#Comprobaci$\X{on}$ de igualdad
print L1 == V1
print L1 == V1.subspace([v1,v1+v2]) 
\end{lstlisting}

\begin{lstlisting}
#Comprobaci$\X{on}$ de inclusi$\X{on}$
print L1 <= V1; L1 >= V1; L1 >= V1.subspace([v1])
\end{lstlisting}


\begin{lstlisting}
#Intersecci$\X{on}$ y suma de subespacios
L1 = V1.subspace([(1,1,0),(0,0,1)])
L2 = V1.subspace([(1,0,1),(0,1,0)])
L3 = L1.intersection(L2)
print '* Intersecci$\texttt{ón}$ de subespacios: '
print L3
L4 = L1+L2
print '* Suma de subespacios: ';L4
\end{lstlisting}

\end{enumerate}
\subsection{Bases y coordenadas}

Como hemos visto, se define un subespacio de un espacio vectorial en {\sage}
mediante un conjunto de generadores del subespacio,   pero internamente se
guarda   mediante una base del subespacio. Esta base, en principio no es un
subconjunto del conjunto generador utilizado para definir el subespacio, pero
podemos imponer que lo sea  con el método \lstinline|subspace_with_basis|:
\begin{lstlisting}
L1 = V1.subspace_with_basis([v1,v2,v3])
print L1.basis(); L1.basis_matrix()
\end{lstlisting}

El método \lstinline|.coordinates()| nos da las coordenadas de un vector en la
base del
espacio
\begin{lstlisting}
print (L1.coordinates(v1), L2.coordinates(v1))
\end{lstlisting}


\subsection{Producto escalar}

Podemos definir un espacio vectorial con una forma bilineal mediante 

\begin{lstlisting}
V = VectorSpace(QQ,2, inner_product_matrix=[[1,2],[2,1]])
\end{lstlisting}

Acabamos con una  lista, incompleta, de funciones y métodos relacionados con el
producto escalar de vectores.

\def\bfitem#1:{\item\lstinline|#1|.- }
\begin{itemize}
\renewcommand{\labelitemi}{$\circ$}
\bfitem u.dot_product(v): producto escalar, $u\cdot v$
\bfitem u.cross_product(v): producto vectorial, $u\times v$
\bfitem u.pairwise_product(v): producto elemento a elemento
\bfitem norm(u), u.norm(), u.norm(2): norma Euclídea
\bfitem u.norm(1): suma de coordenadas en valor absoluto
\bfitem u.norm(Infinity): coordenada con mayor valor absoluto
\bfitem u.inner_product(v): producto escalar utilizando la matriz del producto
escalar
\end{itemize}

\section{Ejercicios}


\subsection{Inducci\'on y sucesiones}
\begin{ejer}
Demuestra por inducci\'on sobre $n\in\mathbb{N}$ las afirmaciones siguientes:
\label{rec}
\begin{enumerate}

\item $1^2+2^2+\cdots+n^2= \dfrac{n(n+1)(2n+1)}6.$

\item $\dfrac{1}{1\cdot 2}+\dfrac{1}{2\cdot 3}+\cdots+\dfrac{1}{n\cdot (n+1)} =
\dfrac{n}{n+1}\,,\text{ si }n\ge1.$
\item $1\cdot 1!+2\cdot 2!+\cdots +n\cdot n!=(n+1)!-1.$

\item $ \dfrac12+\dfrac2{2^2}+\dfrac3{2^3}+\cdots+ \dfrac n{2^n}\,=\,
2-\dfrac{n+2}{2^n}.$
\item $(1+q)(1+q^2)(1+q^4)\cdots (1+q^{2^n}) \,=\, \dfrac{1-q^{2^{n+1}}}{1-q}.$

\end{enumerate}

En este ejercicio se pide una ``demostraci\'on matem\'atica'' por inducci\'on. M\'as adelante \hyperref[induccion]{veremos} c\'omo programar  la comprobaci\'on, hasta un $n$ prefijado, de f\'ormulas copmo estas.


\end{ejer}





\begin{ejer}
La sucesión de Fibonacci, $\{F_n\}$, está definida por medio de la
ley de recurrencia:
$$
F_1=1,\quad F_2=1,\quad F_{n+2}=F_n+F_{n+1}.
$$
Calcular los diez primeros términos de la sucesión y comprobar, para $1\le n\le
10$,  la
siguiente identidad:
$$
F_n\,=\, \frac{\left[\frac{1+\sqrt5}2\right]^n\,-\,
\left[\frac{1-\sqrt5}2\right]^n} {\sqrt5}.
$$

La sucesi\'on de Fibonacci vuelve a aparecer varias veces a lo largo del curso,
sobre todo en la  p\'agina \pageref{fibon} y siguientes. 

\end{ejer}

\pagebreak[2]

\begin{ejer}\label{ej-DemoSumaCubos}
Demostrar por inducción la fórmula para la suma de los primeros $n$ cubos
$$
1^3+2^3+...+n^3=\frac{(n+1)^2\,n^2}{4}\,.
$$
En el ejercicio \ref{ej-SumaCubos} se plantea un método para encontrar esta
fórmula. 
\end{ejer}

\begin{ejer}\setcounter{temp}{0}
Estudiar el límite de las siguientes sucesiones
\small
\newcommand{\linea}[2]{\letra \big\{\frac{#1}{#2}\big\}\,;}
$$
\begin{array}{*4{l@{\quad}}}
 \linea{n^2}{n+2}
 &
 \linea{n^3}{n^3+2n+1}
 &
 \linea{n}{n^2-n-4}
 &
 \linea{\sqrt{2n^2-1}}{n+2}
 \\[10pt]
 \linea{\sqrt{n^3+2n}+n}{n^2+2}
 &
 \linea{\sqrt{n+1}+n^2}{\sqrt{n+2}}
 &
 \linea{(-1)^nn^2}{n^2+2}
 &
 \linea{n+(-1)^n}{n}
 \\[10pt]
 \letra \big\{\big(\frac23\big)^n\big\}\,;
 &
 \letra \big\{\big(\frac53\big)^n\big\}\,;
 &
 \linea{2^n}{4^n+1}
 &
 \linea{3^n+(-2)^n}{3^{n+1}+(-2)^{n+1}}
 \\[10pt]
 \letra\big\{\frac{n}{n+1}-\frac{n+1}{n}\big\}\,;
 &
 \letra \big\{\sqrt{n+1}-\sqrt{n}\big\}\,;
 &
 \multicolumn{2}{l}{\letra
\big\{\frac{1}{n^2}+\frac{2}{n^2}+\dots+\frac{n}{n^2}\big\}\,.}
\end{array}
$$
\normalsize
\end{ejer}
\noindent\textbf{Atención: } la sucesión $\frac{3^n+(-2)^n}{3^{n+1}+(-2)^{n+1}}$
tiene
límite $\frac13$. El siguiente código
\begin{lstlisting}[numbers=none]
var('m')
l(m)=3^m+(-2)^m
ll=l(m)/l(m+1)
ll.limit(m=oo)
\end{lstlisting}
devuelve, incorrectamente, \lstinline[basicstyle=\color{NavyBlue}]|0|, %
pero es f\'acil ayudar al int\'erprete:
\begin{lstlisting}[numbers=none]
var('m')
l(m)=3^m+(-2.0)^m
ll=l(m)/l(m+1)
ll.limit(m=oo)
\end{lstlisting}
\begin{Output}
	1/3
\end{Output}

?`Cu\'al puede ser la explicaci\'on?

\begin{ejer}
Calcular, si existen, los límites de las sucesiones que tienen como término
general
$$
a_n=\Big(\frac{n^2+1}{n^2}\Big)^{2n^2-3}\,,\quad
b_n=\Big(\frac{n^2-1}{n^2}\Big)^{2n^2+3}\,,\quad
c_n=a_n+\frac1{b_n}\,.
$$
\end{ejer}

\subsection{Un ejercicio de c\'alculo}

Consideramos el polinomio en la variable $x$ y dependiente de dos par\'ametros
reales, $a$ y $b$, dado por 
\[p(x,a,b):=x^4-6x^2+ax+b,\]
\noindent y queremos estudiar el n\'umero de ra\'{\i}ces reales que tiene
dependiendo del valor de los par\'ametros.

\begin{enumerate}
 \item Es un teorema importante, llamado en ocasiones el {\itshape teorema
fundamental del \'algebra}, el hecho cierto de que todo polinomio de grado $n$
con coeficientes complejos tiene exactamente $n$ ra\'{\i}ces complejas si se
cuentan con sus multiplicidades.
\item Si un polinomio tiene coeficientes reales entonces tiene un n\'umero par
de ra\'{\i}ces complejas no reales, ya que si un complejo no real es
ra\'{\i}z su complejo conjugado tambi\'en lo es. 
\item En consecuencia, el n\'umero de ra\'{\i}ces reales de un polinomio de
grado cuatro con coeficientes reales es siempre par.

\item Supongamos para empezar que $a=0$, y podemos empezar dibujando la
gr\'afica para diversos valores de $b$. Si no estaba claro desde el principio,
vemos que todas las gr\'aficas son iguales, la joroba doble de un dromedario
invertida, y que al crecer $b$ la gr\'afica ``sube'' respecto a los ejes.

Entonces, para valores suficientemente grandes de $b$ no habr\'a ninguna
ra\'{\i}z real y para valores muy negativos de $b$ habr\'a dos ra\'{\i}ces
reales. Para valores intermedios deber\'{\i}a haber cuatro ra\'ices reales ya
que el eje $OX$ puede cortar a las dos jorobas. 
 
 \item Cuando $a=0$ podemos tratar el problema usando que podemos escribir
 \[p(x,0,b)=(x^2-3)^2-9+b,\]
 \noindent y vemos inmediatamente que si $9-b<0$ no puede haber ra\'{\i}ces
reales y las cuatro ra\'{\i}ces son n\'umeros complejos no reales. Los casos
restantes ($b\le 9$) se pueden tratar de forma similar, resolviendo
expl\'{\i}citamente la ecuaci\'on $p(x,0,b)=0$.

\item Supongamos ahora que $a\ne 0$ y vamos a tratar el caso particular en que
$a=8$. Si observamos cuidadosamente la gr\'afica para $a=8$ y $b=0$ vemos que
una de las jorobas ha desaparecido y la parte de abajo de la gr\'afica se ha
aplanado. ?`Por qu\'e? Cuando estamos cerca del origen  los dos primeros
sumandos de  $p(x,a,b)$ toman valores muy peque\~nos, si $x$ es peque\~no $x^4$
es mucho m\'as peque\~no, y el t\'ermino $ax+b$ domina. 

\item Para ver el motivo de la desaparici\'on de una de las jorobas debemos
calcular los m\'aximos y m\'{\i}nimos de la funci\'on, es decir, debemos derivar
e igualar a cero. Vemos entonces que lo que tiene de especial el valor $a=8$ es
que para ese valor y $x=1$ se anulan la primera y la segunda derivadas. En la
gr\'afica eso se ve como una zona casi horizontal cerca de $x=1$.

\item En la gr\'afica tambi\'en parece verse que la funci\'on $p(x,8,0)$ es
creciente para $x\ge 0$. ?`Es esto verdad?
 
 \item Debemos pensar que la desaparici\'on de una de las jorobas implica que
ahora s\'olo puede haber $2$ ra\'{\i}ces reales o bien ninguna. ?`C\'omo se
puede probar esta afirmaci\'on y para qu\'e valores de $b$ se obtendr\'{\i}a 
cada uno de los casos?

\item ?`Puedes decir algo,  acerca del n\'umero de ra\'{\i}ces reales en
funci\'on de $a$ y $b$, si $a\ne
0$ y $a\ne \pm 8$?
 
\end{enumerate}


Puedes ver una soluci\'on en la hoja 
\href{http://localhost:8888/notebooks/CAVAN/24-CAVAN-raices-reales.ipynb}{\tt 24-CAVAN-raices-reales.ipynb},
y de la parte m\'as matem\'atica en este
\href{http://150.244.21.37/PDFs/CAVAN/un_ejercicio_calculo.pdf}{archivo.}


\subsection{Algunos ejercicios m\'as}

\begin{ejer}
Comenzamos con el semic\'{\i}rculo de radio uno y centro el
origen. Elegimos un \'angulo $0<\theta<\pi/2$ y representamos las rectas
$x=sen(\theta)$ y $x=-cos(\theta)$  de forma que junto con el eje $y=0$ y la
circunferencia encierran una regi\'on que llamamos $A$.

Llamamos $B$ al complemento de $A$ en el semic\'{\i}rculo. {\sc Determinar} el
valor m\'aximo, cuando $\theta$ var\'{\i}a, 
 del cociente $F(\theta):=Area(A)/Area(B)$.
 
\end{ejer}

\begin{ejer}
 Consideramos la par\'abola de ecuaci\'on $y=x^2$ y una
circunferencia de centro $(0,a)$ y tal que es tangente\footnote{Dos curvas son
tangentes en un punto en que se cortan si tienen la misma recta tangente en ese
punto.} a la par\'abola en dos puntos distintos. {\sc Determinar} los valores de
$a$ para los que tal circunferencia existe.
\end{ejer}


\begin{ejer}
Determinar el valor de $m$ que hace que la ecuaci\'on 
\[x^4-(3m+2)x^2+m^2=0\]
\noindent tenga cuatro ra\'{\i}ces reales en progresi\'on aritm\'etica (i.e. las ra\'{\i}ces ser\'{\i}an $x_0,x_0+a,x_0+2a,x_0+3a$).	
\end{ejer}

\begin{ejer}
	Sea $M$ un real muy grande. Demuestra que la ecuaci\'on $Mx=e^x$ tiene una soluci\'on, digamos $w$,  \'unica. Estima el valor de $w$, usando como gu\'{\i}a representaciones gr\'aficas de $f(x):=e^x-Mx$, despu\'es de darle a $M$ un valor suficientemente grande. ?`Podr\'{\i}as estimar el valor de $M$ a partir del que se obtiene la soluci\'on $w$ \'unica??`Qu\'e ocurre para valores de $M$ m\'as peque\~nos?
 \end{ejer}

\begin{ejer} Determina el valor m\'{\i}nimo de la constante $a>1$ tal que siempre que $x\le y$  se verifica 
	\[\frac{a+sen(x)}{a+sen(y)}\le e^{y-x}.\]
	
\end{ejer}
\begin{ejer}
	?`Es mayor $e^{\pi}$ que $\pi^{e}$? A una pregunta as\'{\i} podr\'{\i}amos responder calculando los dos n\'umeros reales, con cierta aproximaci\'on, y resulta que ciertamente es mayor $e^{\pi}$. Sin embargo, se trata de demostrarlo en la forma m\'as convincente posible,  sin usar el  valor aproximado. Dicho de otra manera, nuestro argumento nos debe conducir a una demostraci\'on puramente matem\'atica, para la que podemos usar como apoyo el ordenador (gr\'aficas, derivadas, etc.).
	
	\end{ejer}

Puedes ver algo de ayuda para los primeros dos  ejercicios en este
\href{http://150.244.21.37/PDFs/CAVAN/dos_ejercicios_calculo.pdf}{archivo.} Adem\'as, en la hoja
\href{http://localhost:8888/notebooks/CAVAN/27-CAVAN-tangencias.ipynb}{\tt 27-CAVAN-tangencias.ipynb}
puedes ver una animaci\'on de la soluci\'on del segundo.


\subsection{Ejercicios de Álgebra Lineal}\label{interp1}

Antes de plantear una lista de ejercicios  resolvemos, a modo de
ejemplo, un problema de interpolación. Si bien la solución propuesta dista de
ser la más eficiente,
nos sirve como motivación a la manera de hacer que se pretende en este curso. En
una hoja
de Sage, tanto el enunciado como los comentarios en la solución, deberían
aparecer en
cuadros de texto.

\

\begin{ejer}\label{coef-indet} {\upshape Método de coeficientes indeterminados}.
 Encontrar el polinomio de menor grado cuya gráfica pasa por los puntos
 $$
 (-2, 26), (-1, 4), (1, 8), (2, -2)\,.
 $$ 
\end{ejer}


\begin{sol}
 {Puesto que se tienen $4$ puntos, se considera un polinomio general, de grado
$\le 3$: $P(x)=a_0+a_1x+a_2x^2+a_3x^3$.

Con las coordenadas de los $4$ puntos dados, sustituyendo las abcisas e
igualando a las respectivas ordenadas, se obtiene un sistema lineal de $4$
ecuaciones con $4$ incógnitas (los coeficientes, $a_0,a_1,a_2,a_3$, del
polinomio): 
\begin{align*}
1*a_0-2*a_1+4*a_2-8*a_3&=26&&\text{ punto }(-2,26)\\
1*a_0-1*a_1+1*a_2-1*a_3&=4&&\text{ punto }(-1,4)\\
1*a_0+1*a_1+1*a_2+1*a_3&=8&&\text{ punto }(1,8)\\
1*a_0+2*a_1+4*a_2+8*a_3&=-2&&\text{ punto }(2,-2)
\end{align*}
El hecho de que
los $4$ puntos se encuentren en distintas verticales, asegura que el sistema es
compatible determinado. Su matriz  es 
\begin{equation}\notag
 \begin{pmatrix}
  1&-2&4&-8\\
  1&-1&1&-1\\
  1&1&1&1\\
  1&2&4&8
  \end{pmatrix}
\end{equation}
\noindent con inversa 
\[
\left(\begin{array}{rrrr}
-\frac{1}{6} & \frac{2}{3} & \frac{2}{3} & -\frac{1}{6} \\
\frac{1}{12} & -\frac{2}{3} & \frac{2}{3} & -\frac{1}{12} \\
\frac{1}{6} & -\frac{1}{6} & -\frac{1}{6} & \frac{1}{6} \\
-\frac{1}{12} & \frac{1}{6} & -\frac{1}{6} & \frac{1}{12}
\end{array}\right)
\]

\noindent y la soluci\'on del sistema, obtenida multiplicando la matriz inversa
por el vector columna de t\'erminos independientes,  es el vector $(4,5,2,-3)$,
es decir, el polinomio $4+5x+2x^2-3x^3.$}
\end{sol}

\par
\medskip
\par

\begin{comment}
En la hoja de {\sage}
\href{http://sage.mat.uam.es:8888/home/pub/24/}{\tt 25-CAVAN-AL-hoja3.sws}
puedes ver una soluci\'on de algunos ejercicios de la 
 \href{http://150.244.21.37/PDFs/CAVAN/AL1213-hoja3.pdf}{Hoja 3} del curso de \'Algebra Lineal. 

 De forma similar deb\'eis intentar resolver otros ejercicios de las hojas, y
tambi\'en los que se enuncian a continuaci\'on:
\end{comment}

\begin{ejer}\label{ej-SumaCubos}
La suma de los $n$ primeros enteros positivos, $1+2+\dots+n$, es un polinomio en
$n$
de grado
$2$: $\frac12n^2+\frac12n$. La suma de sus cuadrados, $1^2+2^2+\dots+n^2$, es un
polinomio de grado $3$. En general, fijada la potencia, $k$, la suma
$1^k+2^k+3^k+\dots+n^k$ es un polinomio en $n$ de grado $k+1$ ({\sc ?`por
qu\'e?}).

Encontrar, usando interpolaci\'on como en el ejercicio anterior,   una fórmula
para la suma de los cubos de los primeros $n$ enteros
positivos.\footnote{En el ejercicio \ref{ej-DemoSumaCubos}, se pide
una demostración, para este caso, de la afirmaci\'on de que estas sumas son
polinomios en
el n\'umero de sumandos.}

Podemos comprobar el resultado obtenido mediante las instrucciones
\begin{center}
\lstinline|var('m k');sum(k^3,k,1,m)| 
\end{center}
que devuelve directamente el polinomio
buscado, en $m$.
\end{ejer}

\begin{comment}
\begin{ejer}
Escribir, en las bases estándar, la matriz de la aplicación lineal 
$F:\mathbb{Q}^3\longrightarrow \mathbb{Q}^2$ que tiene como n\'ucleo el plano
$\{x+y-2z=0\}$ y
tal que
$F(1,-1,1)=(-1,1)$.
\end{ejer}

\pagebreak[3]

\begin{ejer}
 Consideremos en $\mathbb{Q}^4$ los  subespacios vectoriales
\[W_1=<(2,1,1,0),\,(1,0,2,-1)>\] y   $W_2$ soluci\'on del sistema de ecuaciones
\[
\begin{aligned}
2x-z&=0\\
x+y-z-t&=0.
\end{aligned}
\]



Se pide calcular:
\begin{itemize}
\item las ecuaciones de $W_1$;
\item una base de $W_2$;
\item las dimensiones de $W_1\cap W_2$ y de $W_1+W_2$.
\end{itemize}
\emph{Sugerencia: } averiguar, para una matriz dada, el uso del m\'etodo
\lstinline|.right_kernel()|.
\end{ejer}
\end{comment}

 
%%\subsection{Un ejercicio m\'as de \'Algebra Lineal}
\begin{ejer}
 Dada una matriz $\mathbf{A}$ de tama\~no $m\times n$ con entradas racionales, 
podemos definir una
funci\'on $\Phi_{\mathbf{A}}$, entre espacios vectoriales de matrices,   
mediante 
  \begin{equation}\notag
   \begin{matrix}
    \mathcal{M}_{n\times
k}&\stackrel{\Phi_{\mathbf{A}}}{\longrightarrow}&\mathcal{M}_{m\times k}\\
    \mathbf{B}&\mapsto&\mathbf{A}\cdot \mathbf{B}, 
   \end{matrix}
\end{equation}
 \noindent con $\mathcal{M}_{n\times k}$ ($\mathcal{M}_{m\times k}$) los
espacios de matrices con $n$ filas y $k$ columnas ($m$ filas y $k$ columnas).
 
 Cuando $k=1$, la aplicaci\'on lineal $\Phi_{\mathbf{A}}$ es bien conocida:  se
trata de la obtenida al multiplicar vectores columna de $n$ filas, colocados a
la derecha de $\mathbf{A}$,  por la propia matriz, y nos referimos a ella como
la {\bf aplicaci\'on lineal asociada} a la matriz $\mathbf{A}$. 
 Sabemos que toda aplicaci\'on lineal $u:\mathbb{Q}^n\to \mathbb{Q}^m$ es la
aplicaci\'on asociada a una matriz $\mathbf{U}$, a la que llamamos {\sc la
matriz} de la aplicaci\'on lineal.
 
 \begin{enumerate}
 
  \item Demostrar (completamente) que $\Phi_{\mathbf{A}}$ es, de hecho,  una
aplicaci\'on lineal.
 
 \item Supongamos que la matriz $\mathbf{A}$ es invertible, y, por tanto,
cuadrada. ?`Ser\'a cierto que $\Phi_{\mathbf{A}}$ es necesariamente biyectiva
(es decir, invertible)? 
 
\item Supongamos que la aplicaci\'on lineal que corresponde a $\mathbf{A}$ {\sc
no} es inyectiva. ?`Es cierto que $\Phi_{\mathbf{A}}$ no puede ser inyectiva?

\item Supongamos que $m<n$, ?`es cierto que si la aplicaci\'on lineal que
corresponde a $\mathbf{A}$ es suprayectiva (es decir, de rango $m$) tambi\'en
ser\'a suprayectiva la aplicaci\'on lineal $\Phi_{\mathbf{A}}$??`Es cierta la
afirmaci\'on rec\'{\i}proca?
 
 \item Supongamos ahora que $\mathbf{A}$ es $2\times 2$. En este caso, la
aplicaci\'on lineal $\Phi_{\mathbf{A}}$ va del espacio
de matrices $2\times n$ en s\'{\i} mismo.
 
 \begin{enumerate}
 \item \label{phi-matriz}?`Podr\'{\i}as calcular la matriz de
$\Phi_{\mathbf{A}}$ en las bases
est\'andar de los espacios de matrices (las matrices de esas bases tienen todas
sus entradas $0$ menos una que vale $1$)?

\item Determina el n\'ucleo de $\Phi_{\mathbf{A}}$, y discute los casos
$rango(\mathbf{A})=0,1,2.$ 

 \item Calcula una base del n\'ucleo de $\Phi_{\mathbf{A}}$ en cada uno de los
tres casos.

\end{enumerate}
 \end{enumerate}
\end{ejer}

\begin{comment}
En la hoja de {\sage} 
\href{http://sage.mat.uam.es:8888/home/pub/25/}{\tt 26-CAVAN-phisubA.sws}
puedes ver la soluci\'on de un caso particular del apartado \ref{phi-matriz}.
C\'alculos parecidos,  usando {\sage},  te pueden ayudar bastante a resolver el
ejercicio completo.
\end{comment}
